\documentclass[]{report}
\usepackage[parfill]{parskip}

% Title Page
\title{Plan van Aaanpak Codeniacs platform}
\author{Timo Strating - medewerker bij UU-games}
\date{Februari 13, 2017, Groningen}

\renewcommand*\contentsname{Table of Content}
\pagestyle{headings}

\begin{document}
\maketitle

\tableofcontents
\newpage






\chapter{Achtergrond}

Codeniacs is een interactieve platform met een gepersonaliseerde AI, speciaal ontwikkeld voor kinderen van 8 tot 14 jaar. Het systeem is ontworpen om individueel les te geven aan onze gebruikers. Met dit platform willen wij spelenderwijs kinderen stimuleren om te gaan leren programmeren. Het Tijdschrift van Codeniacs speelt een grote rol in de gameplay van de verschillende spellen en opdrachten die op het platform te vinden zijn. Ons platform is interactief en zit boordevol nieuwe features zoals AI en virtual reality features.

\section{Relatie met andere initiatieven}
Codeniacs is het initiatief van Stichting Eclipse Incubators Foundation ( Stichting EIF). De stichting heeft een groot netwerk en zal deze inzetten om promotie te cre\"{e}ren voor Codeniacs. CoderDojo is een organisatie dat zich bezig houdt met kinderen leren programmeren. Zij doen dit onder andere door middel van een programma genaamd Scratch. Scratch is een webbased programmeer omgeving voor kinderen.




\chapter{Project Opdracht}

\section{Codeniacs beschrijving}
Codeniacs bestaat uit 5 onderdelen. Als eerste zal Codeniacs een tijdschriften serie uitbrengen. In deze serie worden verwijzingen gemaakt naar het platform. De tijdschriften geven handvatten aan de doelgroep om aan de slag te gaan met de applicaties die te vinden zijn op het platform. Elke kwartaal wordt er een tijdschrift gepubliceerd van 24 pagina's.

Op aanvulling van de tijdschriften wordt twee maal per jaar een boek uitgegeven. Elk boek heeft een eigen thema. De boeken puilen uit van mooie illustraties, striptekeningen, opdrachten, QR-codes, virtual reality markers, links naar websites en meer. Alles is erop gericht om kinderen aan het programmeren te krijgen.

De applicaties op het platform bestaan uit serious games. De mini games verschillen per tijdschrift, boek en gebruiker. Zie gamedesign document voor meer informatie over de applicaties en zie platform beschrijving voor een uitgebreide uitleg over het platform.
Naast het maken van applicaties voor de doelgroep bieden wij ook zelfgemaakte video's aan waarin informatie komt te staan over de nieuwste ontwikkelingen op ICT gebied, instructies voor de applicaties, vlogs van onze werknemers, uitleg over het lesmateriaal en nog veel meer.

Onze Codeniacs website heeft een inlog pagina voor de doelgroep. Als extra bieden wij in de toekomst een speciale platform voor ouders en leraren. Op dit platform staat onder andere lesmateriaal en uitleg, docentenhandleidingen en een vraag en antwoord forum. Zo kunnen ouders en docenten die niet ICT opgeleid zijn toch hun klas voorzien van ICT gerelateerde opdrachten. In het design moet hier dus rekening mee gehouden worden. Dit deel zal dan in een toekomstige versie ontwikkeld worden.


\section{Platform beschrijving}
Codeniacs is een interactieve platform met een gepersonaliseerde AI. Het systeem is ontworpen om individueel les te geven aan onze gebruikers. Dit doen wij door middel van een gids. De gids zal altijd in beeld zijn en interactie vormen met de gebruiker. Alle informatie dat de gebruiker geeft wordt opgeslagen in het “geheugen” van de gids en wordt gebruikt om hulp te bieden wanneer de gebruiker vastloopt. De gids zal in het begin een grote rol spelen. De gebruiker kan kiezen om uitleg te krijgen over de UI van het platform en de verschillende applicaties hierop. Zie het kopje Gids voor meer informatie.

De input van de klant zorgt ervoor dat het hele systeem gepersonaliseerd wordt. Stel je voor dat wij 1 kind uit onze doelgroep halen en kennis laten maken met ons platform. Een paar kernwoorden dat dit kind typt in ons platform zijn; geschiedenis, honden, vakantie, basketbal en animatie films. In dit geval zal het platform zich personaliseren met onderdelen die te maken hebben met deze onderwerpen.

Alle applicaties krijgen trefwoorden, mochten er applicaties gemaakt zijn waarin de gekozen onderwerpen voorkomen dan zullen die applicaties worden aangeboden aan het kind. De trefwoorden worden opgeslagen, zo weten wij wat de vraag is van onze doelgroep en kunnen wij hierop inspelen met onze applicaties en weetjes.

Daarnaast kunnen de gebruikers zelf bepalen hoe Codeniacs er uit ziet, welke gids ze willen tijdens het gebruik van ons platform en welk niveau applicaties er aangeboden wordt. De bedoeling is dat de doelgroep groeit en dat Codeniacs mee groeit met de individuele behoeftes van de gebruiker.


\section{Project beschrijving}
Dit platform zal een koppeling maken naar het Codeniacs tijdschrift dat geschreven wordt door Wim en geïllustreerd wordt door Peter. Het platform en de tijdschriften zijn gemaakt voor kinderen tussen de 8 en 14 jaar. Het platform zal bij coderdojo's getest worden.

\section{Gids}
Tijdens het eerste gebruik van het platform wordt een gids gekozen. De gids is een karakter dat uit het Codeniacs kinderboekenserie komt. Alle karakters hebben verschillende eigenschappen en deze eigenschappen zullen tijdens het gebruik van het platform ook naar voren komen.

Kiest de gebruiker bijvoorbeeld voor Flo de kat dan zal de kat af en toe kattenkwaad uithalen als de gebruiker op het platform zit. Flo is een wijs katje dus zal ook wijze woorden gebruiken. Woorden die gebruikt worden door de kat worden uitgelegd als de gebruiker met de muis op het woord hangt. Mocht het woord uit onze eigen begrippen pagina komen en klikt de gebruiker daarop, dan wordt het woord via een interactieve manier uitgelegd. Denk hierbij aan een mini game, een visual novel of een animatie video.

De Gids is een slim ontworpen systeem dat ervoor zorgt dat de gebruiker altijd handvatten krijgt toegereikt indien nodig. Het moet als “vriendje” dienen voor de gebruiker tijdens de kennismaking met het programma. 




\chapter{Project Activiteiten}

\begin{itemize}
	\item Face 1: Ontwerp - (17 dagen)
	\begin{enumerate}
		\item De vraagstelling en/of informatiebehoefte wordt vastgesteld
		\item Plan van Aanpak (PvA)
		\item Functioneel ontwerp (FO)
		\item Technisch ontwerp (TO)
		\item Toelichting ontwikkel omgeving
		\item Definition of done (DoD)
		\item TestRapport 
		\item Acceptatietest
		\item Het op zetten van de ontwikkel omgeving
		\item Het opzetten van een Git repo
		\item Overleg omtrent het starten van het project
		\item De documentatie eventueel aanpassen naar wensen van de klant of behartigden
			\newline
	\end{enumerate} 
	
	
	\item Face 2: Realisatie - (12 dagen)
	\begin{enumerate}
		\item Het opzetten van de Git master branche repo pipeline
		\item Het bouwen van een ontwikkel pipeline
		\item Het bouwen van kleine tests en doedels om garanties over de implementatie te kunnen geven
			\newline
	\end{enumerate} 
	
	
	\item Face 3: Inplementatie  - (9 dagen)
	\begin{enumerate}
		\item Het ontwikkelen van de front-end
		\item Het ontwikkelen van de back-end
		\item Het Onderhouden van een stabiele versies
		\item Het uitbrengen van nieuwe stabiele versies
		\item De andere Codeniacs projecten de toegang geven tot de aangevraagde features 
			\newline
	\end{enumerate} 


	\item Face 4: Onderhoud  - (12 dagen)
	\begin{enumerate}
		\item Feedback verwerking
		\item Feature request verwerking
		\item Het verbeteren van de kwaliteit van de code
		\item Het verder stabiliseren van de uiteindelijke versie
	\end{enumerate} 
\end{itemize} 





\chapter{Project grenzen}

\section{Lengte van het project}
Er zal 10 weken 4 dagen in de week worden gewerkt aan het project.

\section{Breedte van het project}
Er moet rekening worden gehouden dat het Platform een centrale rol gaat spelen in de verbonden projecten van Codeniacs.

\section{Deadline}
Het project zal doorgaan tot de afgesproken doelen zijn behaald en/of afgehandeld binnen de afgesproken deadline deze deadline is op 5 Mei 2017.

De deadline voor de opdracht is gezet op begin mei. Het gaat om over een schatting van 12 weken waarin 4 dagen in de week van 09.00 uur t/m 17.00 uur gewerkt wordt aan de opdracht. Nadat het project is afgelopen en het eindproduct is opgeleverd is het ontwikkel team niet meer verantwoordelijk voor het eindproduct en onderhoud.





\chapter{Producten}

\begin{itemize}
	\item Plan van Aanpak (PvA)
	\item Functioneel ontwerp (FO)
	\item Technisch ontwerp (TO)
	\item Toelichting ontwikkel omgeving
	\item Definition of done (DoD)
	\item TestRapport 
	\item Acceptatietest
		\newline
	\item Private Github repository
	\item Development pipeline 
	\item Development enviorment pipeline
	\item Automatische tests
	\item Webbased Platform werkend op een locale ontwikkel omgeving
		\newline
	\item Testgebruikers resultaten documentatie
		\newline
\end{itemize} 





\chapter{Kwaliteit}

\section{Inleiding}
Om de kwaliteit te waarborgen beginnen we eerst met een technisch ontwerp. De opdrachtgever zal het technisch ontwerp te zien krijgen voor het wordt gerealiseerd. Samen met de opdrachtgever zal uiteindelijk een Project worden gecreëerd volgens de afspraken van de beide partners. 

\section{Tussenproducten}
Dit idee zal worden gerealiseerd tot eindproduct. Ook zullen er tussenproducten zijn. Die tussenproducten worden met de project groep besproken om tot een goed eindproduct te komen. Hiernaast worden de guidelines van de pipeline gevolgd dit wil zeggen dat er geen groter releases naar buiten gebracht mag worden zonder dat alle geautomatiseerde test slagen

\section{Eindproduct}	
Als aan alle eisen is voldaan van zowel de opdrachtgever als de projectgroep zal het product als eindproduct worden beschouwd. In het technisch ontwerp staan alle stappen beschreven hoe het eindproduct moet werken.

\section{Controle}	
Om de kwaliteit te waarborgen zullen wij tussentijds de gemaakte onderdelen testen. Dit zullen wij doen door op verschillende apparaten het product te testen. Mochten er complicaties optreden op een ander apparaat zullen wij het veranderen totdat het werkt. Als het eindproduct af is zullen wij deze uitgebreid gaan testen. Deze testen zullen wij doen op verschillende apparaten door verschillende mensen. Het idee is dat alle mogelijkheden meerdere keren getest gaan worden. 

\section{Code}		
Alle code zal worden ontwikkeld volgens de regels opgesteld in de DoD (devinition of done). Daarnaast zal de opgezette pipelines er voor zorgen dat er altijd een goed inzicht in de kwaliteit van de code.





\chapter{Projectorganisatie}

\section{codeniacs team}
\begin{tabular}{ l l l }
	\textbf{Naam} & \textbf{Functie} & \textbf{Email adres} \\ \hline
	Garlon Hashams 			& Verloop begeleider				& Helpdesk@incubatorsfoundation.com \\ 
	Naomi Versluis 			& Ontwikkelspecialist 				& Nversluis@uu-games.com \\
	Martijn de Vries 		& Game designer – Programmeur 		& Aduarder@hotmail.com \\
	Peter Bekkema			& Illustrator						& Bekkema64@gmail.com \\
	Timo Strating 			& Programmeur 						& timo@snowbyte.nl \\
	Andrew Livsey 			& Schrijver							& andrew@textralab.nl \\
	Ruurd van der Weide		& Schrijver							& ruurd@textralab.nl \\
\end{tabular}


\section{ontwikkel team}
\begin{center}
	Timo Strating - oprichter van \textbf{snowbyte.nl} en stagiaire bij \textbf{UU-games}
\end{center}	

\begin{center}
	\begin{tabular}{ l c }
		Bedrijf & AlsJeDitZietStuurDanGewoonEenMailtje@snowbyte.nl \\
		Persoonlijk & timostrating@home.nl  \\
		Student & tj.strating@student.alfa-college.nl \\
	\end{tabular}
\end{center}


\section{Informatie / communicatie}

Alle benodigde informatie die het ontwikkel team nodig heeft zal op worden geleverd door de opdrachtgever naarmate het project draaiend is. 

Communicatie tussen het ontwikkelteam en de opdrachtgever zullen verlopen via face to face meetings, skype meetings, telefonisch contact, berichtjes via Whatsapp, Skype berichtjes en mail verkeer.




\chapter{Planning}

\begin{itemize}
	\item Face 1: Ontwerp - (17 dagen)	
	\item Face 2: Realisatie - (12 dagen)	
	\item Face 3: Implementatie  - (9 dagen)	
	\item Face 4: Onderhoud  - (12 dagen)
		\newline
\end{itemize} 

Zie Project Activiteiten voor meer informatie




\chapter{Kosten en baten}

Er zijn geen directe kosten verbonden aan dit project. Een vrijwillige bijdrage uit dankbaarheid wordt wel gewaardeerd de de project ontwikkelaar.




\chapter{Risico’s}

Omtrent de project ontwikkelaars werkplek / ruimte is nog niet alles duidelijk. Hierdoor zijn er tijdelijke oplossingen naar voren gebracht door het bedrijf. Dit kan daarom dus voor mogelijke vertraging opleveren in de realisatie van het project.



\end{document}      
