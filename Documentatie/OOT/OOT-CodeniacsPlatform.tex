\documentclass[]{report}
\usepackage[parfill]{parskip}

\usepackage{tikz}
\usetikzlibrary{arrows.meta}

% Title Page
\title{Ontwikkel Omgeving Toelichting Codeniacs platform}
\author{Timo Strating - medewerker bij UU-games}
\date{Februari 23, 2017, Groningen}

\renewcommand*\contentsname{Table of Content}
\pagestyle{headings}

\begin{document}
\maketitle

\tableofcontents
\newpage






\chapter{Inleiding}

Dit project zal vanuit 0 uit worden gewerkt naar een 2.0 uitbreidbare versie. Dit zorgt er voor dat er ook meerdere pipelines opgesteld zullen tijdens de ontwikkeling van het project. 

Omdat er vanuit 0 gestart word zijn er nog veel onduidelijkheden. Tijdens de ontwikkeling is er daarom een vergrote kans dat elementen binnen het systeem vervangen zullen worden voor andere elementen. 

Zodra er bijvoorbeeld een nieuwe tool wordt ge\"{\i}ntroduceerd tijdens dit proces zal dit worden opgenomen in dit document.





\chapter{Hardware}

\begin{tabular}{ l p{8cm} }
	\textbf{Hardware} & \textbf{Vereisten} \\ \hline
	Computer	&	1 gigahertz (GHz) of sneller 64-bits (x64) processor \\
	 & 4 GB RAM \\
	 & 32 GB beschikbare schijfruimte \\
	 & 64 bits Windows OS, 8.1 of 10 \\
	Internettoegang	 &	Wifi of ethernet \\
\end{tabular}




\chapter{Software}

\section{Inleiding}
Dit is een standaard lijst met items die gebruikt zullen worden door het ontwikkel team. Alle producten die hier onder genoteerd staan zullen nodig zijn of het meest effici\"{e}nt zijn. Programma's kunnen dus vanvangen worden door persoonlijke voorkeuren maar de lijst hier onderaan geeft inegeval weer wat werkt. 

Iedere ontwikkelaar dient verbonden te zijn aan de Github repository met een persoonlijk account dat de rechten toegekend gekregen heeft gekregen.


\section{Server}
\begin{tabular}{ l l p{6cm} }
	\textbf{Software} & \textbf{Versie} & \textbf{Opmerking} \\ \hline
	Node.js & ~6.9.5 & Any stable release around this version should work.\\
	MongoDB & 3.4 \\
	NPM & 3.10.10  \\
	Meteor & 1.3.1  \\
\end{tabular}


\section{Development}
\begin{tabular}{ l l p{6cm} }
	\textbf{Software} & \textbf{Versie} & \textbf{Opmerking} \\ \hline
	Node.js & 6.9.5  \\
	MongoDB & 3.4 \\
	NPM & 3.10.10  \\
	Meteor & 1.3.1  \\
	Yeoman & 1.0+ & Dit moet geinstaleed worden via NPM. Dit is alleen in de eerste face van het project nodig. \\
	\\
	Google Chrome & 55+ \\
	Mozilla Firefox & 50+ \\
	Microsoft Edge & 14+ \\
	\\
	WebStorm & 9.0+ & Is niet een gratis IDE. \\ 
	Atom & 1.14+ & Goede vervanger voor WebStom maar het is niet een IDE. \\
	Nodepad ++ & 7.0+ \\
	\\
	Koala & 2.0+ & Handig niet verplicht. \\ 
	Git Bash & 2.10+ & Handig niet verplicht. \\ 
	Github Desktop & 3.3.3+ & Voor he geval je niet met de git cli overweg kan .\\ 
	\\
	MiKTeX & 2.9+ & Benodigd voor TexStudio. \\ 
	LyX & 2.2+ & GUI version of TexStudio. \\
	TexStudio & 2.12+ & We adviseren om de Nederlandse opensource woordenboeken van Libre office toe te voegen voor betere vertalingen. \\
\end{tabular}






\end{document}      
